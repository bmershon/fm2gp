\documentclass{article}
\usepackage{lmodern}
\usepackage{amsmath}
\usepackage{hyperref}
\hypersetup{
    colorlinks=true,
    linkcolor=blue,
    filecolor=magenta,      
    urlcolor=blue,
}

\title{From Mathematics to Generic Programming}
\author{Brooks Mershon}
\date{March 2017}

\begin{document}

\maketitle

\section*{6.9}

\textit{Solution.}

We have a group with 101 distinct elements. Lagrange's Theorem tells us that \textit{the order of any subgroup $H$ in a finite group $G$ divides the order of the group}. Well, \href{https://bl.ocks.org/bmershon/8bed98a4633d86403e1ca56165cda6da}{101 happens to be prime}. So the only subgroups that can exist must have orders that are divisors of 101, otherwise we will have contradicted Lagrange's Theorem. The divisors of $101$ are $1$ and $101$; the subgroups of G are $\left\{ 1 \right\}$ and $G$, the trivial subgroups.

\end{document}
