\documentclass{article}
\usepackage{lmodern}
\usepackage{amsmath}
\usepackage{hyperref}
\hypersetup{
    colorlinks=true,
    linkcolor=blue,
    filecolor=magenta,      
    urlcolor=blue,
}

\title{From Mathematics to Generic Programming}
\author{Brooks Mershon}
\date{March 2017}

\begin{document}

\maketitle

\section*{6.10}

\textit{Solution.}

Any group $G$ of prime order $p$ must have as its subgroups $\left\{ 1 \right\}$ and $G$, otherwise we would find a contradiction with Lagrange's Theorem: \textit{the order of any subgroup $H$ of a finite group $G$ divides the order of that group}.

Let $a$ be an element other than the identity element in $G$. The order of $a$ must be $p = \lvert G \rvert$, by the above reasoning. The order of the identity element is $1$, and that is the only element which can have order $1$ (otherwise that other element must be the identity element). If the order of an arbitrary $a$ in $G$ is the order of the group itself, then we must be able to find some $m \geq 0$ such that $b = a^m$ for any $b \in G$. That is, G is a cyclic group. We would not be able to prove this if we couldn't draw upon Lagrange's Theorem and our knowledge of the primality of the order of $G$; if the order of $G$ were not prime, we could not guarantee that $a$ would not have an order less than $n$, and further, we could not guarantee that some $a$ exists for which every $b \in G$ is reachable by some power of $a$.


\end{document}
