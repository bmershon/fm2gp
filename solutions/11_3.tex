\documentclass{article}
\usepackage{lmodern}
\usepackage{amsmath}
\usepackage{hyperref}

\title{From Mathematics to Generic Programming}
\author{Brooks Mershon}
\date{April 2017}

\begin{document}

\maketitle

\section*{11.3}


\textit{Solution.}

\bigskip

When $n = 2$, we see that any element of $S_n$ is a transposition or the identity permutation  (do nothing). So commutativity trivially follows. When $n > 2$, we might intuitively suspect that permutations are not commutative because functions are, in general, not commutative: $f(g(x)) \neq g(f(x))$. Intuitively, pure functions depend on their inputs, so swapping functions around in terms of composition seems like it would, in general, not preserve inputs for the respective functions. 

Consider $s_1$, $s_2$, and $s_3$ as elements in $S_n$ where $n > 3$. Let $\alpha$ be the permutation which transposes $s_1$ and $s_2$ and let $s\beta$ be the permutation which transposes $s_2$ and $s_3$.

$$\alpha \circ \beta = (3\ \ 1\ \ 2)$$

$$\beta \circ \alpha = (2\ \ 3\ \ 1) \neq \alpha \circ \beta$$

We see that permutations which \textit{intersect} will tend to not allow for commutativity, as this counterexample shows. $S_n$ is not abelian for $n > 3$.


\end{document}
