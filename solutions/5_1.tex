\documentclass{article}
\usepackage{lmodern}
\usepackage{amsmath}

\title{From Mathematics to Generic Programming}
\author{Brooks Mershon}
\date{March 2017}

\begin{document}

\maketitle

\section*{5.1}

\textit{Solution.}

Let us first expand the factorial:

$$(n - 1)! = (n - 1)(n - 2)(n - 3) \dotsm (n - (n - 1)) $$

We know $n$ is composite, which means we can either find two positive integers $a$ and $b$, both less than n which when multiplied together give us  $n$, \textit{or} we may find n is a composite number equal to $a^2$.

In the first case, we can simply choose the numbers $a$ and $b$ from the set $\left\{n-1, n-2, n-3, \dots, 1\right\}$. These two integers do appear as factors in the expanded factorial, so we know (n - 1)! is a multiple of n.

In the latter case, we need two ``copies'' of some factor $a$. What to do, since we only have a set of integers in the interval $\left[1, n - 1\right]$? Since we are looking to show that $(n - 1)! = mn$ for some multiple $m$ of $n$, we can (when $n > 4$) go ahead and just multiply $a$ by $2a$ in order to obtain a multiple of $n$ among the factors multiplied together in the expanded factorial. We are guaranteed when $n > 4$ that the factor $2$ exists for us to choose from when forming $2a$.

In both cases we have produced a multiple of $n$ (or simply $n$ itself) and then expect to multiply this multiple by whatever other factors were not used in our explicit construction.

\end{document}
