\documentclass{article}
\usepackage{lmodern}
\usepackage{amsmath}

\title{From Mathematics to Generic Programming}
\author{Brooks Mershon}
\date{March 2017}

\begin{document}

\maketitle

\section*{6.4}

\textit{Solution.}

The order of $e$ is $1$ because the order of an element $a$ of some group is defined as the smallest $m$ such that $a^m = e$. Well, $e^1 = e$. $e0$ is not necessarily equal to $e$ (what could it equal?), and the definition (6.7) of the \textbf{order} of an element happens to explicitly define this value for values of $m > 0$.

Now, suppose some other element $a_0$ satisfied

$$a_0^1 = e.$$

Then by the definition of raising a group to a power, $a_0^1 = a_0 = e$. So $e$ is the only element of order $1$.

\end{document}
