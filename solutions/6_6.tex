\documentclass{article}
\usepackage{lmodern}
\usepackage{amsmath}
\usepackage{hyperref}
\hypersetup{
    colorlinks=true,
    linkcolor=blue,
    filecolor=magenta,      
    urlcolor=blue,
}

\title{From Mathematics to Generic Programming}
\author{Brooks Mershon}
\date{March 2017}

\begin{document}

\maketitle

\section*{6.5}

\textit{Solution.}

We can refer to the table of nonzero remainders modulo 7 and simply trace the steps taken from repeated applications of a particular element $a$. The first application is simply $a$ itself, so if we determine that $a \cdot a \cdot a = 1 = e$, then we shall say the order of $a$ is $3$.

\begin{center}
\begin{tabular}{ c|c|c|c|c|c|c|c} 
 & \textbf{1} & \textbf{2} & \textbf{3} & \textbf{4} & \textbf{5} & \textbf{6} & \textit{order} \\
 \hline
 \textbf{1} & 1 & 2 & 3 & 4 & 5 & 6 & 1 \\ 
 \hline
 \textbf{2} & 2 & 4 & 6 & 1 & 5 & 5 & 3 \\ 
 \hline
 \textbf{3} & 3 & 6 & 2 & 5 & q & 4 & 6\\ 
 \hline
 \textbf{4} & 4 & 1 & 5 & 2 & 6 & 3 & 3\\ 
 \hline
 \textbf{5} & 5 & 3 & 1 & 6 & 4 & 2 & 6\\
 \hline
 \textbf{6} & 6 & 5 & 4 & 3 & 2 & 1 & 2\\
 \hline
\end{tabular}
\end{center}

For example, we have the following progression of repeated applications of the element $5$:

$$5 \rightarrow 4 \rightarrow 6 \rightarrow 2 \rightarrow 3 \rightarrow 1.$$

There are $5$ arrows, but we must include $5^1$ as the ``first'' application of $5$; the order of the element $5$ is $6$. Note that the only element with order 1 is the identity element $1$.

A larger value of $n$ involves more labor if we choose to continue finding solutions by hand. We might consider a programmatic solution to finding orders of elements in a multiplicative group modulo $n$. The following example renders a table like the one seen above for $n = 11$. One can see how such a solution lends itself to a generalized solution that obviates the need to use pencil and paper to find the orders of, say, $n = 101$. Of course, even a fairly humble algorithm like the one seen in the implementation below requires a bit of thinking to get right.

\bigskip

\url{https://bl.ocks.org/bmershon/7938f064dc2202364cdd52acbd24805d}

\end{document}
