\documentclass{article}
\usepackage{lmodern}
\usepackage{amsmath}

\title{From Mathematics to Generic Programming}
\author{Brooks Mershon}
\date{March 2017}

\begin{document}

\maketitle

\section*{6.7}

\textit{Solution.}

We recall that a cyclic group $G$ has an element $a$ such that for any element $b$, there is an integer $n$ where

$$b = a^n$$

Now, if we have a subgroup $H$ which is the singleton set $\left\{ 1 \right\}$, then this trivial subgroup of $G$ is generated by its only element.

If instead we have a nontrivial subgroup $H$, then let $m$ be the smallest positive integer such that $a^m \in G$ (this excludes considering the identity element.

Now consider an arbitrary element $b \in H$. We know $b = a^n \in G$ for some $n$, since H is a subgroup of G. We can certainly say that $n = l \cdot m + r$ for integers $l$ and $r$, where $r$ is the remainder when we attempt to divide $n$ by $m$. So $0 \leq r \geq m$. What we shall do is use the assumption that $m$ is the smallest integer such that $a^m \in G$ along with some algebraic manipulation of our expression for $n$ in order to show that $a^m \in H$ is a generating element which may be used to generate an arbitrary $b \in H$ by raising $a^m$ to some power.

With some substitution and rearranging, we have

\begin{align*}
    a^n &= a^{lm + r} = \left(a^m\right)^l a^r
\end{align*}

This much we achieved by substituting and observing properties of exponents. Now for some more rearrangement:


\begin{align*}
    a^r &= \frac{a^n}{\left( m \right)^l} = a^{n - ml} = \left(a^{-m}\right)^l a^n \\
        &=  \left(a^m \right)^{-l} a^n
\end{align*}

Now, we know that since $H$ is a subgroup, the inverse of $a^m$ must be in $H$. And because a group is closed under powers, we know powers of the inverse of $a^m$ must be in $H$; therefore, $\left(a^m\right)^{-l} \in H$. But our arbitrary $b = a^n$ is also in $H$. Under the closure property of groups, $\left(a^m \right)^{-l} a^n$ must be in $H$ as well.

Since we have constructed our remainder $r$ such that $0 \leq r \geq m$, and we assumed that $m$ was the smallest positive integer such that $a^m$ is in $H$. This implies $r = 0$, otherwise we would contradict our assumption about $m$.

Immediately then, we have that $n = lm$ and therefore

$$b = a^n = a^{lm} = \left(a^m\right)^l $$

Since we let $b$ be an arbitrary element in $H$, we know any $b$ in $H$ may be generated by an element in $H$ which we know can be constructed by raising it to some power $l$. So \textit{any} subgroup of a cyclic group is cyclic.

\end{document}
